\documentclass[10pt, twocolumn, twoside, letterpaper]{IEEEtran}
\IEEEoverridecommandlockouts

\usepackage[activate={true, nocompatibility}, final, tracking=true, kerning=true, spacing=true, factor=1100, stretch=10, shrink=10]{microtype}
% \linespread{0.92}

\makeatletter
\def\mycopyrightnotice{
  {\footnotesize
  \begin{minipage}{\textwidth}
  \centering
  978-1-6654-8872-3/22/\$31.00 \copyright2022 IEEE
  \end{minipage}
  }
}
\def\ps@IEEEtitlepagestyle{
  \def\@oddfoot{\mycopyrightnotice}
  \def\@evenfoot{}
}

\ifCLASSINFOpdf
   \usepackage[pdftex]{graphicx}
\else
   \usepackage[dvips]{graphicx}
\fi

\ifCLASSOPTIONcompsoc
  \usepackage[caption=false, font=normalsize, labelfont=sf, textfont=sf]{subfig}
\else
  \usepackage[caption=false, font=footnotesize]{subfig}
\fi

\usepackage{amsmath}
\usepackage{bm}
\usepackage{amssymb}
\usepackage{algorithm}
\usepackage{algorithmic}
\usepackage{stfloats}
\usepackage{url}
\usepackage{siunitx}
\usepackage{fancyref}

\usepackage{geometry}
\geometry{letterpaper, top=0.7in, bottom=0.7in, left=0.65in, right=0.65in}

\input{glossary}

\usepackage[style=ieee, doi=false, isbn=false, url=false, maxbibnames=1, minbibnames=1, maxcitenames=1, mincitenames=1, backend=biber, defernumbers=false]{biblatex}
\addbibresource{./bibtex/bib/Biblio.bib}
\addbibresource{./bibtex/bib/extrabib.bib}

% \AtEveryBibitem{\clearfield{month}}
% \AtEveryBibitem{\clearfield{day}}
% \AtEveryBibitem{\clearfield{volume}}
% \AtEveryBibitem{\clearfield{issue}}
% \AtEveryBibitem{\clearfield{pages}}
% \AtEveryBibitem{\clearfield{number}}
% \AtEveryBibitem{\clearfield{title}}
% \AtEveryBibitem{\clearfield{isbn}}
% \AtEveryBibitem{\clearfield{keywords}}
% \AtEveryBibitem{\clearfield{issn}}
% \AtEveryBibitem{\clearfield{journal}}

\begin{document}

\title{
    % \vspace{-0.75cm}

    Pseudo-Bayesian DIP Denoising as a Preprocessing Step for Kinetic Modelling in Dynamic PET
}

\pagestyle{plain}
\pagenumbering{gobble}

\author{
    % \vspace{-0.4cm}
    
    \IEEEauthorblockN{
        Alexander~C.~Whitehead~\IEEEauthorrefmark{1}~\IEEEauthorrefmark{2},
        Kjell~Erlandsson~\IEEEauthorrefmark{1},
        Ander~Biguri~\IEEEauthorrefmark{3},
        Scott~D.~Wollenweber~\IEEEauthorrefmark{4},
        Jamie~R.~McClelland~\IEEEauthorrefmark{2} and
        Kris~Thielemans~\IEEEauthorrefmark{1}~\IEEEauthorrefmark{2}
    }
        
    \IEEEauthorblockA{
        \IEEEauthorrefmark{1}
        \textit{Institute of Nuclear Medicine, University College London, UK}
    }
    \IEEEauthorblockA{
        \IEEEauthorrefmark{2}
        \textit{Centre for Medical Image Computing, University College London, UK}
    }
    \IEEEauthorblockA{
        \IEEEauthorrefmark{3}
        \textit{Department of Applied Mathematics and Theoretical Physics, University of Cambridge, UK}
    }
    \IEEEauthorblockA{
        \IEEEauthorrefmark{4}
        \textit{Molecular Imaging and Computed Tomography Engineering, GE Healthcare, USA}
        % \vspace{-0.75cm}
    }

    \thanks{Manuscript received November 19, 2022.}
    \thanks{This research was supported by GE Healthcare, the NIHR UCLH Biomedical Research Centre and the UCL EPSRC Centre for Doctoral Training in Intelligent, Integrated Imaging in Healthcare (i4health) grant (EP/L016478/1).}
    \thanks{The software used was partly produced by the Computational Collaborative Project on Synergistic Biomedical Imaging, CCP SyneRBI, UK EPSRC grant (EP/T026693/1).}
    \thanks{Jamie~R.~McClelland is supported by a Cancer Research UK Centres Network Accelerator Award grant (A21993) to the ART-NET consortium and a CRUK Multi-disciplinary grant (CRC 521).}
    \thanks{(contact: \texttt{alexander.whitehead.18@ucl.ac.uk}).}
}

\maketitle
\IEEEpeerreviewmaketitle

\begin{abstract}
    Noise (among other artefacts) could be considered to be the bane of PET. %Often, it causes what could otherwise be a more simple problem to explode in complexity. 
    Many methods have been proposed to alleviate the worst annoyances of noise, however, not many take into account the temporal nature of dynamically acquired PET. Here, we propose an adaption of a method, which has seen increasing attention in more traditional imaging denoising circles. Deep Image Prior exploits the initialisation of a carefully designed neural network, so as to treat it as a bank of custom filters, which are to be trained and used afresh on each new image, independently. Deep Image Prior has seen adaptation to PET previously (including dynamic PET), however, many of these adaptations do not take into account the large memory requirements of the method. Additionally, most previous work does not address the main weakness of the Deep Image Prior, its stopping criteria. Here, we propose a method which is both memory efficient, and includes a smoothing regularisation. In addition, we provide uncertainty estimates by incorporating a Bayesian approximation (using dropout), and prototype a training scheme by which the model is fit on all data simultaneously. The denoised images are then used as input for kinetic modelling. To evaluate the method, dynamic XCAT simulations have been produced, with a field of view of the lung and liver. The results of the new methods (along with total variation and the old Deep Image Prior) have been compared by; a visual analysis, SSIM, and $K_i$ values. Results indicate that the new methods potentially outperform the old methods, without increasing computation time, while reducing system requirements.
\end{abstract}

% \vspace{-0.5cm}

\section{Introduction} \label{sec:introduction}
    \IEEEPARstart{M}{ost} \gls{ML} or \gls{NN} based methods, rely upon a workflow where a model is designed, trained, validated, and deployed~\cite{Krose2011AnNetworks}.% This is logical as the intention with \gls{NN} based methods is to treat them as a general function approximator~\cite{Krose2011AnNetworks}. This is where the function is determined by the relationship between the data used and the objective or loss function. 
    However, in the domain of image denoising, the \gls{DIP} method has received attention due to its promising results, and the fact that training and inference are performed independently on each new image~\cite{Ulyanov2018DeepPrior}. The \gls{DIP} method could be considered to be a custom learnt bank of filters for each input. In order to prevent overfitting, the number of iterations used is imperative.% The authors of the original \gls{DIP} paper argue that the method could be considered to be used to solve many inverse problems~\cite{Ulyanov2018DeepPrior}.
    
    For \gls{PET}, there have been a number of adaptations of \gls{DIP}, for instance;~\cite{Gong2019PETPrior} represented one of the first applications of \gls{DIP} to \gls{PET}. Here, a U-Net is used, with relatively high count/low motion brain scans~\cite{Weng2015U-Net:Segmentation}. In~\cite{Hashimoto20214DNetwork} \gls{DIP} is extended to \gls{4D} dynamic \gls{PET}. To do this, multiple output branches are grafted onto the \gls{NN}, one for each dynamic time point.~\cite{Hashimoto2019DynamicDatasets} uses the original or static \gls{PET} acquisition as input to the \gls{NN}, rather than noise.~\cite{Yang2022SimultaneousPrior} represents a more recent extension of \gls{DIP}, where multiple \glss{NN} are used simultaneously.
    
    This work seeks to extend or simplify previous work, in order to give a solution, which can be used to denoise \gls{4D} dynamic \gls{PET} data, for the application of kinetic modelling. Firstly, we reduced \gls{GPU} memory requirements. Because \gls{DIP} requires training at inference, the full amount of memory is needed every time the method is used. To aid in clinical adoption, a hard limit of \SI{8.0}{\giga\byte} \gls{GPU} memory was imposed. Secondly, a more robust stopping criteria is required.% The stopping criteria should be flexible so as to allow the method to correct a wider variety of inputs. 
    One method would be, to look at a window of previous loss function values, and exit, when the gradient of this drops below a tolerance. To aid in achieving this, as well as to address weaknesses of the original \gls{DIP}, regularisation must be added, so as to stop the \gls{NN} fitting to noise. Previously, \gls{TV} has been incorporated in generic \gls{DIP} literature~\cite{Liu2019ImagePrior}, but has not been adopted in \gls{PET} \gls{DIP}. Finally, because the output from this method is to be used in a further model fitting, it may provide improved results to have a metric of the uncertainty in the denoised images. Most ways to incorporate Bayesian inference include, either increasing memory usage, or execution time. However, in~\cite{Gal2015DropoutLearning} it is proposed to use dropout to approximate Bayesian inference. Furthermore, this work uses \gls{PET} data with a \gls{FOV} of the lung and liver, whereas most previous work using a \gls{FOV} of the head.

% \vspace{-0.5cm}

\section{Methods} \label{sec:methods}
    \subsection{\gls{PET} Acquisition Simulation and Image Reconstruction} \label{sec:pet_acquisition_simulation_and_image_reconstruction}
        A series of dynamic scans, following the clinical \gls{DWB}-\gls{PET} protocol, were generated using the \gls{XCAT} phantom~\cite{segars4DXCATPhantom2010}. Assigned patient-derived kinetic parameters to, $64$ tissues, three tumours of \SI{1.0}{\centi\meter} diameter in the left lung, and three tumours of \SI{2.5}{\centi\meter}, \SI{2.0}{\centi\meter} and \SI{1.0}{\centi\meter} diameter in the liver. An input function for \gls{18F-FDG}, taken from \cite{langsjoEffectsSubanestheticKetamine2004}, was used to simulate \glss{TAC} to create dynamic images.
% KT some info here on timing

        \gls{PET} acquisitions were simulated (and reconstructed) using \gls{STIR}~\cite{Thielemans2012} through \gls{SIRF}~\cite{Ovtchinnikov2017}. \gls{NTOF} sinogram data were simulated, using resolution modelling (using a \SI{6.0}{\centi\meter} \gls{FWHM} Gaussian filter). Randoms and scatter were not included. Poisson noise was added corresponding to a total number of counts of XXX in time frame XXX.
% KT some info above

        Finally, all data sets were reconstructed using $10$ iterations with $17$ subsets of \gls{OSEM}~\cite{Hudson1994}.

    % \vspace{-0.5cm}
    
    \subsection{Network Design and Execution} \label{sec:network_design_and_execution}
        The \gls{NN} used was a modified U-Net~\cite{Weng2015U-Net:Segmentation}, with seven down/upsampling stages. Each down/upsampling stage consisted of two convolutional layers (with two, four, eight, $16$, $32$, $64$, or $128$ channels, depending on depth). Followed by, either a split strided convolution and maxpooling layer (with the result concatenated), or a trilinear upsampling layer. Edge padding, group normalisation~\cite{Wu2018GroupNormalization}, MISH activation~\cite{Misra2020Mish:Function}, and spatial dropout, were used with every convolutional layer. Data was edge padded to the nearest power of two, and the input data had Gaussian noised summed to it. Both input and label data were standardised. \gls{MSE} and \gls{TV} were used as the loss function. AdamW~\cite{Loshchilov2017DecoupledRegularization} was used as optimiser. Training continued for all methods until the gradient of the loss function, over a window of previous results, reduced below a threshold. Parameters were tuned using a grid search.
        
        Two training regimes were explored, one where each time point was treated independently, and another, where the model weights were saved and then independently updated on each time point (the mean of the new models weights was taken for the next iteration).
        
    % \vspace{-0.5cm}
    
    \subsection{Kinetic Modelling} \label{sec:kinetic_modelling}
        Indirect Patlak estimation was used to generate $K_i$ and intercept images \cite{patlak1983GraphicalEvaluationBloodtoBrain}. Uncertainty estimates were used XXX.
        
    % \vspace{-0.5cm}
    
    \subsection{Evaluation} \label{sec:evaluation}
        In addition to the denoising performed above%in~\Fref{sec:network_design_and_execution}
        , data were also denoised using, \gls{TV}, and the \gls{DIP} method presented in~\cite{Gong2019PETPrior}.
        
        Comparisons used included: A visual analysis, \gls{SSIM} to the ground truth~\cite{Wang2009MeanMeasures}, a \gls{TAC} through a lesion, a profile over a lesion, and \gls{SUV}\textsubscript{max} and \gls{SUV}\textsubscript{peak} (defined following \gls{EANM} guidelines~\cite{Boellaard2015FDG2.0}).

% \vspace{-0.5cm}

\section{Results} \label{sec:results}
    \begin{figure}
        % \vspace{-0.5cm}
        
        \centering
    
        \includegraphics[width=1.0\linewidth]{figures/visual_analysis.png}    
        % \includegraphics[width=1.0\linewidth]{figures/visual_analysis.png}
        
        % \vspace{-0.5cm}
        
        \captionsetup{singlelinecheck=false, justification=centering}
        \caption{
        % \scriptsize
        A visual analysis between the ground truth and denoised results (plus \gls{SSIM} to the ground truth), for; the ground truth, the original noisy data, and this data denoised using, \gls{TV}, the implementation of \gls{DIP} from~\cite{Gong2019PETPrior}, and our new implementation of \gls{DIP}, trained sequentially and combined (taken for the last time point of the lung \gls{FOV}). Colour map ranges are consistent for all images.}
        
        \label{fig:visual_analysis}
        
        % \vspace{-0.5cm}
    \end{figure}
    
    \begin{figure}
        % \vspace{-0.5cm}
        
        \centering
    
        \includegraphics[width=1.0\linewidth]{figures/tac.png}    
        % \includegraphics[width=1.0\linewidth]{figures/tac.png}
        
        % \vspace{-0.5cm}
        
        \captionsetup{singlelinecheck=false, justification=centering}
        \caption{
        % \scriptsize
        A \gls{TAC} through a lesion, fit as a third order polynomial regression, with weighting using uncertainty (where available), for; the ground truth, the original noisy data, and this data denoised using, \gls{TV}, the implementation of \gls{DIP} from~\cite{Gong2019PETPrior}, and our new implementation of \gls{DIP}, trained sequentially and combined, both with and without uncertainty (taken for the lung \gls{FOV}).}
        
        \label{fig:tac}
        
        % \vspace{-0.5cm}
    \end{figure}
    
        \begin{figure}
        % \vspace{-0.5cm}
        
        \centering
    
        \includegraphics[width=1.0\linewidth]{figures/profile.png}    
        % \includegraphics[width=1.0\linewidth]{figures/profile.png}
        
        % \vspace{-0.5cm}
        
        \captionsetup{singlelinecheck=false, justification=centering}
        \caption{
        % \scriptsize
        A profile through a lesion, in the \gls{SI} direction, for; the ground truth, the original noisy data, and this data denoised using, \gls{TV}, the implementation of \gls{DIP} from~\cite{Gong2019PETPrior}, and our new implementation of \gls{DIP}, trained sequentially and combined (taken for the last time point of the lung \gls{FOV}).}
        
        \label{fig:profile}
        
        % \vspace{-0.5cm}
    \end{figure}
    
    \begin{table}
        % \vspace{-0.5cm}
        
        \centering
        
        \captionsetup{singlelinecheck=false, justification=centering}
        \caption{
        % \tiny
        Comparison of \acrshort{SUV}\textsubscript{max} and \acrshort{SUV}\textsubscript{peak}, for; the ground truth, the original noisy data, and this data denoised using, \gls{TV}, the implementation of \gls{DIP} from~\cite{Gong2019PETPrior}, and our new implementation of \gls{DIP}, trained sequentially and combined (taken for the last time point of the lung \gls{FOV}).}
        
        % \vspace{-0.5cm}
        
        \resizebox*{1.0\linewidth}{!}
        % \resizebox*{1.0\linewidth}{!}
        {
            \begin{tabular}{||c|cc||}
                \hline
                \textbf{\acrshort{SUV}}             & \textbf{Max}  & \textbf{Peak} \\
                \hline
                \textbf{Ground Truth}               & $12.3$        & $9.53$ \\
                \hline
                \textbf{Noisy}                      & $21.5$        & $5.99$ \\
                \hline
                \textbf{\gls{TV}}                   & $3.28$        & $6.24$ \\
                \textbf{Original \gls{DIP}}         & $3.90$        & $6.93$ \\
                \hline
                \textbf{New \gls{DIP} Sequential}   & $9.19$        & $8.02$ \\
                \textbf{New \gls{DIP} Combined}     & $9.27$        & $8.21$ \\
                \hline
            \end{tabular}
        }
        \label{tab:suv}
        
        % \vspace{-0.5cm}
    \end{table}
    
    A visual comparison of the reconstructed images (see~\Fref{fig:visual_analysis}), shows that both of the new \gls{DIP} methods perform comparably, if not for a slight reduction in noise in the combined case. Whereas, the \gls{TV} and original \gls{DIP} implementations appear to have struggled with, over smoothing, reducing the contrast of the lesions, and introducing some edge artefacts.
    
    The overall shape of the \gls{TAC} (see~\Fref{fig:tac}) for the new \gls{DIP} combined method appears, most similar to the ground truth, however, with a slight reduction in quantification. The sequential method is less accurate, but still more so than both the original \gls{DIP} and \gls{TV} methods. There is significant variation in the noise \gls{TAC}, somewhat masked by the regression, however, its shape is still least like the ground truth. Adding uncertainty appears to have improved the \gls{TAC} of the new \gls{DIP} sequential method, however, the uncertainty of the combined method is less, and as such, the inclusion of uncertainty has not affected results significantly.
    
    The peak of the profile (see~\Fref{fig:profile}) for both new \gls{DIP} methods is comparable, and greater than both the original \gls{DIP} and \gls{TV} methods. The peak of the noise profile is greater than all other methods, including the ground truth, however, this is not necessarily beneficial, as can be seen by the rest of the profile not closely following the ground truth (it undulates unpredictably). The profile for both new \gls{DIP} methods are significantly smoother, and more closely follow the ground truth.
     
    \acrshort{SUV} (and \acrshort{SSIM}) results confirm the above (see~\Fref{tab:suv}).

% \vspace{-0.5cm}

\section{Discussion and Conclusions} \label{sec:discussion_and_conclusions}
    % XXXSome discussion of results here.

    Results presented here were obtained on a single bed position. Initial evaluation on a bed position, centred on the liver, indicated that parameter fine-tuning, depending on the distribution and count level, will be beneficial. Evaluation with patient data will follow.
    
    The uncertainty estimates produced by the \gls{NN} need to be validated by comparison with results obtained from repeated noise realisations.
    
    % In the future, research will focus on the application of the method to domains other than dynamic \gls{PET}, where \gls{4D} data exists, such as \gls{MC}.

% \AtNextBibliography{
    % \scriptsize
%     \small
% }
\printbibliography

\end{document}
